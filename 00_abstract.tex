
\begin{jabstract}

スマートフォンやパーソナルコンピュータの急激な普及につれ,様々な応用ソフト(アプリケーション)が開発され人々が毎日利用している.近年ではアプリケーション開発において十分な開発時間,人員,技術が確保することが難しく質の担保されたアプリケーションを設計し開発することが難しくなっている.そして個人開発者,スタートアップ企業,アイディアはあるが実装する力のない個人などではさらにこれを行うことが難しくなる.

その流れに伴い「ノーコード」と呼ばれるプログラミングやコンピュータサイエンスに精通していない人でも簡単にアプリケーションを開発できるサービスが登場した.しかしながら現在のノーコードにおいてもアプリケーションのユーザインタフェースデザインはユーザが行うか,テンプレートを利用する他ないためユーザインタフェースに精通していないユーザでは自由度の高く質の担保されたユーザインタフェースのアプリケーションを開発するのは困難である.

そこで本研究では,本来大規模な開発でしか行われない専門のユーザインタフェースデザインを自動化し,ユーザインタフェースに関する専門的な知識がなくても容易にユーザインタフェースを開発できるようにするためのシステムの開発をおこなった.ユーザインタフェースの設計の方法については確立された手法はなくデザイナのセンスと経験に委ねられていた部分が多かったが,既存のユーザインタフェースを分析していくことでユーザインタフェースは画面に表示されている要素の数と種類,各要素のグルーピング,画面内での要素がもつ優先度が決定されればインターフェースを制作できるのではないかという仮説を立てた.その仮説に基づきiOS向けのネイティブアプリケーションのユーザインタフェースを専門的知識なしに制作できるシステムを開発し有用性を明らかにした.
\end{jabstract}


\begin{eabstract}
High-quality, easy-to-use UI design requires in-depth professional knowledge and experience.
We will propose a system that builds UI automatically.  Our system enables designing UI without professional knowledge and experience.

There is no established method for UI design.
So, it has often been left to the sense and experience of the designer. 

    However, the UI of smartphone applications is similar and has not changed significantly for decades. 
Hence, we thought that there must be some kind of rule.

By analyzing existing interfaces, we hypothesized that  UI can be built if the number and types of elements, the grouping of each element, and the priority of each element are determined.
Also, in general, when we talk about "design" in software, we tend to focus on the appearance of the interface.

By automating the appearance of interfaces, a designer can focus on the essential design. 
 Based on this hypothesis, we developed a system that allows users to generate user interface source codes for native iOS applications (SwiftUI) without specialized knowledge.  
After that, evaluated the usefulness of this system.

\end{eabstract}
