
\begin{jabstract}

スマートフォンやパーソナルコンピュータの急激な普及につれ,様々な応用ソフト(アプリケーション)が開発され人々が毎日利用している.近年ではアプリケーション開発において十分な開発時間,人員,技術が確保することが難しく質の担保されたアプリケーションを設計し開発することが難しくなっている.そして個人開発者,スタートアップ企業,アイディアはあるが実装する力のない個人などではさらにこれを行うことが難しくなる.

その流れに伴い「ノーコード」と呼ばれるプログラミングやコンピュータサイエンスに精通していない人でも簡単にアプリケーションを開発できるサービスが登場した.しかしながら現在のノーコードにおいてもアプリケーションのユーザインタフェースデザインはユーザが行うか,テンプレートを利用する他ないためユーザインタフェースに精通していないユーザでは自由度の高く質の担保されたユーザインタフェースのアプリケーションを開発するのは困難である.

そこで本研究では,本来大規模な開発でしか行われない専門のユーザインタフェースデザインを自動化し,ユーザインタフェースに関する専門的な知識がなくても容易にユーザインタフェースを開発できるようにするためのシステムの開発をおこなった.ユーザインタフェースの設計の方法については確立された手法はなくデザイナのセンスと経験に委ねられていた部分が多かったが,既存のユーザインタフェースを分析していくことでユーザインタフェースは画面に表示されている要素の数と種類,各要素のグルーピング,画面内での要素がもつ優先度が決定されればインターフェースを制作できるのではないかという仮説を立てた.その仮説に基づきiOS向けのネイティブアプリケーションのユーザインタフェースを専門的知識なしに制作できるシステムを開発し有用性を明らかにした.
\end{jabstract}


\begin{eabstract}
With the rapid spread of smart phones and personal computers, a variety of application software have been developed and are used by people every day. In recent years, it has become difficult to secure sufficient development time, manpower, and technology to design and develop applications with guaranteed quality. This is even more difficult for individual developers, start-up companies, and individuals who have ideas but lack the ability to implement them.

In response to this trend, a service called "no-code" has emerged, which allows people who are not familiar with programming or computer science to easily develop applications. However, even with the current "no-code" service, the user interface design of the application can only be done by the user or by using a template, making it difficult for users who are not familiar with user interfaces to develop applications with a high degree of freedom and guaranteed quality. Therefore, it is difficult for users who are not familiar with user interfaces to develop applications with high flexibility and quality.

Therefore, in this research, we developed a system to automate specialized user interface design, which is normally done only in large-scale development, so that users without specialized knowledge of user interfaces can easily develop user interfaces. There is no established method for designing user interfaces, and much of the work has been left to the designer's sense and experience. However, by analyzing existing user interfaces, we hypothesized that a user interface can be created if the number and type of elements displayed on the screen, the grouping of each element, and the priority of each element within the screen are determined. Based on this hypothesis, we developed a system that can create user interfaces for native applications for iOS without specialized knowledge, and clarified its usefulness.

\end{eabstract}
