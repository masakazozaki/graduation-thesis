\chapter{結論}
\label{chap:conclusion}
前章まででUIの表層的なデザインは画面内の要素,その優先度,グルーピングで行うことができるのではないかという仮説を立てた.
そして仮説をもとにシステムを実装し,既存UIの再現,ユーザテストを経て本システムの有用性を示すことができた.
本研究により現在では大規模な開発でしか行われていない専門性の高いユーザインタフェースデザインのハードルを下げ,表層のビジュアルデザインに捉われずに,奥深くのデザインに注力する手法を示すことができた.


\section{今後の課題}

要素,優先度,グルーピングからUIを自動生成するシステムの開発に成功したものの,現状の本システムはアプリケーションとしての完成度はとても低く,今後の統合的なシステムとして完成させる必要がある.以下に今後の課題と展望を述べる.
\subsection{要素,優先度,グルーピングの入力UIの検討と開発}
本研究におけるUI自動生成システムではユーザが直接システムを使うことはできず,用紙に要素,優先度,グルーピングを記述してもらい,筆者がシステムが受け付けるコードに変換することで実現している.

今後は要素,優先度,グルーピングを効果的に直感的に入力できるインタフェースの研究開発が必要であり今後の修士課程で,研究を進めていきたい.
\subsection{画面遷移を司るUIの自動生成}
本研究では画面のUIを画面遷移をつかさどるUIと画面のコンテンツを表示するUIの二種類に分類して自動生成をおこなった.
画面遷移を司るUIに関しても法則性を導き出し自動生成を行えるようにしていきたい.これについても今後の修士課程で研究を進めていきたい.
\subsection{配色の自動生成}
実際のUI設計開発では配色はとても重要であり,色によってアプリケーションの見た目や,サービスのブランディングにも大きく影響してくる.

本研究の自動生成システムではiOSアプリでは最もベーシックな白を基調としブルーをアクセントカラーとするデザインが採用されている.ここに関してもサービスのブランディング,ユーザビリティに適した色を提案するシステムを提案したい.これについても今後の修士課程で研究を進めていきたい.

\subsection{スライダーUIによる表層デザインの調整}
本研究における自動生成システムではUIの表層的な見た目は一意に決定されてしまい,自由度がない.プロトタイプやβ版のアプリケーションであれば問題はないが,リリースするプロダクトに本システムを用いるためにはUIをある程度調整できることが不可欠である.これをAdobe Lightroomに習い,スライダーでアプリケーション全ての画面を一括に直感的に変更するシステムも現在研究中であり,本研究のシステムと統合することでより自由度の高く,使いやすいデザインをデザインの専門知識なくとも実現することができる.こちらに関しても修士課程で研究を進めていきたい.

\subsection{統合的なシステムとしての展望}
本研究で開発したUI自動生成システムを基盤とし,前述の課題点を解決するアルゴリズムを開発し,macOS向けのネイティブアプリケーションでの開発を予定している.以下の図\ref{fig:autogen-integrated},図\ref{fig:autogen-integrated2}は現状の統合的なシステムのモックアップである.

\begin{figure}[htbp]
  \begin{minipage}{\hsize}
    \begin{center}
       \includegraphics[width=100mm]{img/autogen-integrated.png}
    \end{center}
    \caption{統合デザインシステムのプロトタイプ}
    \label{fig:autogen-integrated}
  \end{minipage}
\end{figure}


\begin{figure}[htbp]
  \begin{minipage}{\hsize}
    \begin{center}
       \includegraphics[width=100mm]{img/autogen-integrated2.png}
    \end{center}
    \caption{統合デザインシステムのプロトタイプ2}
    \label{fig:autogen-integrated2}
  \end{minipage}
\end{figure}

%軽く最後に書く.2章には持ってこない.
\chapter{関連研究と諸概念の整理}
\label{chap:prevresearch}

\section{ユーザインタフェースのデザイン手法}

%なんか内容がびみょういな.知りたいのはそういうことではない
%今UIデザインは大変だ 配置とかが大変だ

\section{ノーコードでのシステム開発}
大規模な開発を行うリソースがない企業や専門知識をあまり持たない個人でもアプリケーションやウェブサイトを作れるように近年,ノーコードと呼ばれるサービスが登場した.ノーコードとは文字通り,コーディングを行わずにアプリケーションやウェブサイトを作れるサービスである.

全体を通して,コードを書かないのでコードを書けばできることができない.


ここでは既存のノーコードのサービスの例をいくつかあげ,特徴と課題点を挙げていく.

\subsection{MIT App Inventor}
MIT App Inventor\cite{mitappinventor}はGoogleが開発し,現在はMIT Media Labが運営する,スマートフォンアプリケーションのノーコードシステムである.MIT AppInventorでは専門性のない人でもandroidアプリをコードを書かずに開発することができる.コードは書かなくていいものの,コードを書く作業をGUI化しているためとても複雑で使いこなすのは難しい上,UI設計に関しては完全にユーザに委ねられており,質の高いUIを構築するにはUIに関する専門知識が求められる.
また,androidアプリのソースコードとして書き出すことができないため,App Inventorでできることの範囲を超えてしまった場合,アプリケーションをandroid studioで作り直す必要がある.
\subsection{STUDIO}
STUDIOはSTUDIO株式会社が運営する

デザインの自由度が高いのがウリ.自由度の高いノーコードってイラレみたいで結局表層のビジュアルデザインに目を向けがち.
web front-endの専門知識はなくても行えるが,デザインの専門知識は求められる.
自由度の高いビジュアルデザインを売りにしているが,細かなインタラクションなどを追加でソースコードを書くことで補えないのでSTUDIOのできる範囲内で行わなければならない.
素人ではなくある程度知識のある人が使うものであるにもかかわらず,STUDIOで制作したwebサイトは一目でわかってしまうため,コードを書けないと自分で宣言しているようなものである.

\subsection{Glide}
Google Spread Sheetsなどをデータベースとして扱い,テンプレートベースの簡易なアプリを作れる.
スケーラビリティが終わっている
テンプレートベース
社内ツールの自動化や小規模のもの向けでこれで作ったものをプロダクトとして出すことは難しい


\subsection{Microsoft Power Apps}
Glideと同じような感じであるが,若干のスケーラビリティがる.
同じくビジネスツールなのでそれ自体をプロダクトにすることは難しい

\section{デザインの自動化}
デザインの専門知識がない人でもテンプレートに頼らず適切のUIを制作できるようにUIの自動生成を行う研究が行われている
\subsection{機械学習を用いたUIの自動生成}
機械学習使うと次同じインプットでやった時にも同じのが出てくると限らないので適していない
\subsubsection{GANを用いた自動生成}
\cite{2021guigan}
\subsubsection{GPT-3を用いた自動生成}
\cite{text2app}
