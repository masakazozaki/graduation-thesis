\chapter{序論}
\label{chap:introduction}

\section{背景}

アプリケーション開発が近年盛んになってる。

デザインの重要性を述べる
%Goodpatchを持ってくる。シリコンバレーは共同創業者にデザイナがいてユーザの体験をきちんと設計していた

そしてデザインは表層のデザインとされがちであるが本質は奥深くにあると述べる
%Goodpatchのダイアグラムを引用
本システムでは勘違いされやすい表層のデザインを自動化することで
だが

\subsection{大規模開発に置けるユーザインタフェースデザイン}
\subsection{個人開発に置けるユーザインタフェースデザイン}
実開発で費やされるUIデザインの時間について述べる

(1)現在は大規模な開発でしか行われていない専門性の高いユーザインタフェースデザインのハードルを下げること,(2)表層のデザインにとらわれずに奥深くのデザインに注力することができる手法を開発することが求められる.

\section{目的}
(1),(2) を満たすための,自動化で要素,優先度,グルーピングを使ったものを開発した.そして,開発したシステムの有用性を示した.
%
\section{本論文の構成}

本論文の構成を示す.

第\ref{chap:introduction}章では本研究の背景について述べた.第\ref{chap:prevresearch}章では関連研究と諸概念を整理する.第\ref{chap:pulsewave}章では要素,優先度,グルーピングからUIを自動生成する手法を提案し,第\ref{chap:pulsewave}章ではその手法をもとにiOS向けネイティブアプリケーションのコードを生成するシステムを提案する.そして,それらの有効性を示す.最後に,第\ref{chap:conclusion}章の結論では本研究を総括し,考察と展望を述べる.付録として,本研究で行った実験で得られたデータを添付する.
