\chapter{iOSネイティブアプリケーションのインタフェース自動生成システムの開発}
\label{chap:impl}
前述の既存UIの分析と分析から導き出したアルゴリズムをもとにiOSのネイティブアプリケーションのインタフェース自動生成システムの開発を行った.

また,普段からiOSのネイティブアプリケーションを個人開発している14-17歳の男女を被験者とした本システムの有用性をはかる実験を行い,有用性を示すことができた.

本システムの設計は以下のとおりである.
\section{システムの設計・開発}
要素,優先度,グルーピング情報からの画面のコンテンツUI自動生成システムのプロトタイプとして主要な部分のみを実装した.本研究において画面のコンテンツUIとは実際画面に表示されるUIのうち,画面遷移を司どるUI,画面のタイトル表示を除いたものを指す.
本研究のUI自動生成において重要なのは画面内の要素,優先度,グルーピング,の情報のみでUIを自動生成できることである.この点を実現するためのシステムを開発した.

\subsection{基本的なレイアウトアルゴリズム}
前述のアルゴリズムより,基本的には要素を縦に順に配列する.グループも一要素として扱いレイアウトを行う.グループ内のレイアウトについては御述のグルーピングセクションで行う.
基本的な優先度

\subsection{優先度}


\subsection{グルーピング}
\subsubsection{グルーピングの入れ子構造}

\subsection{表示レイアウトアルゴリズム}

\section{システムの有効性についての実験}

\subsection{対象と手続き}

\subsection{結果と考察}