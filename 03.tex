\chapter{ノーデザイン: 要素,優先度,グルーピングの情報からのUI自動生成手法の提案}
\label{chap:auto-gen}

本章では,後述のiOSネイティブアプリケーションのインタフェース自動生成システムの開発に先立ち,既存のインタフェースの分析から導きさした要素,優先度,グルーピングによりインタフェースの構築が行える法則について検討した.自動生成を行うアルゴリズムの提案行う.
\section{既存インタフェースの分析}
既存のiOSネイティブアプリケーションのUIの分析をおこなっていく.
画面内でのUIは画面遷移を司り,画面内の体験には直接影響を与えないものが存在する.
わかりやすくするために,画面遷移を司るUIとそうでないUIについて分類して考えていく.
\subsection{画面遷移を司るUI}
HIGのhierarchical Navigation,
FlatNavigation
Content-Driven or Experience-Driven Navigation
大体こうなってる.


今回の自動生成手法の提案ではこのような画面遷移を司るUIは切り離し,画面内のコンテンツのUIに注目して考えいく.
実際のSwiftUIのコードでも画面遷移を司るUIとコンテンツは別で実装されて画面遷移の上にコンテンツが乗る形式になる.のでこの方針は問題ない



\subsection{要素}
画面内に何を置くかは考える必要がある
要素をマーキングしたスクショを何枚か載せる
\subsection{優先度}
優先度がありそうだ.
優先度も考えたダイアグラムを何枚か載せる
\subsection{グルーピング}
グルーピングのダイアグラムを何枚か載せる
グルーピングがありそうだとのべる

\section{アルゴリズムの構築}
要素を優先度の高い順に画面内に縦に並べていくのが基本.
グループは一つの要素としてカウントする.(優先度は中の要素の優先度の足し合わせ)
要素の中にはユーザが操作するもの,見るだけのものがあり,操作するものは全体に並ぶと良い.
Groupが入れ子になる場合はHstack, Vstackを交互にすることによって大体のUIを構成できる.
画面内の要素の優先度の基準を10とし,それを超える場合はスクロールするようにする.
これで大体いい感じになる.Padding等は適切につけていく.
優先度が上がる度に文字は大きく,太く,ボタンもただの青文字ではなく,borderがついたり,塗り領域が増えたりとしていく.

