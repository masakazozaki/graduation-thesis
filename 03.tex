
%軽く最後に書く.2章には持ってこない.
\chapter{関連研究と関連プロダクトの整理}
\label{chap:prevresearch}

\section{ノーコードでのシステム開発}
大規模な開発を行うリソースがない企業や専門知識をあまり持たない個人でもアプリケーションやウェブサイトを作れるように近年,ノーコードと呼ばれるサービスが登場した.ノーコードとは文字通り,コーディングを行わずにアプリケーションやウェブサイトを作れるサービスである.
ここでは既存のノーコードのサービスの例をいくつかあげ,特徴と課題点を挙げていく.

\subsection{MIT App Inventor}
MIT App Inventor\cite{mitappinventor}はGoogleが開発し,現在はMIT Media Labが運営する,スマートフォンアプリケーションのノーコードシステムである.MIT AppInventorでは専門性のない人でもandroidアプリをコードを書かずに開発することができる.コードは書かなくていいものの,コードを書く作業をGUI化しているためとても複雑で使いこなすのは難しい上,UI設計に関しては完全にユーザに委ねられており,質の高いUIを構築するにはUIに関する専門知識が求められる.
また,androidアプリのソースコードとして書き出すことができないため,App Inventorでできることの範囲を超えてしまった場合,アプリケーションをandroid studioで作り直す必要がある.
\subsection{STUDIO}
STUDIOはSTUDIO株式会社が運営するノーコードでWebデザインを行い,リリースできるツールである.
デザインの自由度が高いのが特徴で,web front-endの専門知識はなくてもWebデザインが行えるが,デザインの専門知識は求められる.
自由度の高いビジュアルデザインを売りにしているが,細かなインタラクションなどを追加でソースコードを書くことで補えないのでSTUDIOのできる範囲内で行わなければならない.
素人ではなくある程度知識のある人が使うものであるにもかかわらず,STUDIOで制作したwebサイトは一目でわかってしまうため,サイトの製作者はコードを書けないと閲覧者に印象付けてしまう.

\subsection{Glide}
Google Spread Sheetsなどをデータベースとして扱い,テンプレートベースの簡易なアプリを作れる.
社内ツールの自動化や小規模向けでGlideで作ったアプリケーションをプロダクトとしてリリースすることは難しい.

\section{デザインの自動化}
デザインの専門知識がない人でもテンプレートに頼らず適切のUIを制作できるようにUIの自動生成を行う研究が行われている
\subsection{機械学習を用いたUIの自動生成}
\subsubsection{GANを用いた自動生成}
Tianming ZhaoらのGUIGAN\cite{2021guigan}では敵対生成ネットワークを用いたアプローチでUIの自動生成の手法が提案されている.機械学習を用いたシステムでは出力されるUIを一意的に決定できないほか,利用者がシステムのアルゴリズムを推測して微調整を行うことが難しく,自由度が下がってしまう.
\subsubsection{GPT-3を用いた自動生成}
Masum Hasanらが開発したText2App\cite{text2app}ではGPT-3を用いて自然言語からMIT App Inventorのコードを生成するシステムである.
自然言語でアプリの詳細を伝えるのは難しく,複雑なアプリケーションの開発には向いていない.

