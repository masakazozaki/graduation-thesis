\chapter{ノーデザイン: 要素,優先度,グルーピングの情報からのUI自動生成手法の提案}
\label{chap:auto-gen}

本章では,後述のiOSネイティブアプリケーションのインタフェース自動生成システムの開発に先立ち,既存のインタフェースの分析から導きさした要素,優先度,グルーピングによりインタフェースの構築が行える法則について検討した.自動生成を行うアルゴリズムの提案行う.
\section{既存インタフェースの分析}
既存のiOSネイティブアプリケーションのUIの分析をおこなっていく.
画面内でのUIは画面遷移を司り,画面内の体験には直接影響を与えないものが存在する.
わかりやすくするために,画面遷移を司るUIとそうでないUIについて分類して考えていく.
\subsection{画面遷移を司るUI}
HIGのhierarchical Navigation,
FlatNavigation
Content-Driven or Experience-Driven Navigation
大体こうなってる.


今回の自動生成手法の提案ではこのような画面遷移を司るUIは切り離し,画面内のコンテンツのUIに注目して考えいく.
実際のSwiftUIのコードでも画面遷移を司るUI,画面のタイトルとコンテンツは別で実装されて画面遷移の上にコンテンツが乗る形式になる.のでこの方針は問題ない


\subsection{画面の内容を表示するUI}
前述では画面内のUIについて画面の内容を表示するUIを画面遷移を司るUIが内包していると述べた.本セクションでは画面の内容を表示するUIのレイアウトは主に画面内に表示する要素(UIパーツ),その要素同士の優先度,そして要素同士のグルーピングの3つのルールに則って構成されており,そのルールさえ示せば画面を構成できるのではないかという仮説を立てた.次にレイアウトを構成する3つのルールについて述べる.
\subsubsection{要素}
画面内にどの要素がいくつ存在するのかを定める必要がある.
以下の図では既存のInstagramの画面における要素の抽出をおこなった.
\subsubsection{優先度}
画面内の要素の種類と個数がわかったら次に画面内でその要素の重要度を明確にする.
例えば,SNSサービスであっても写真をメインとするInstagramと文字をメインとするTwitterでは文字と画像の優先度の比重が変わってくる.
以下の図ではInstagramの画面における要素同士の優先度を抽出したものである.

\subsubsection{グルーピング}
要素,優先度に加えて各要素同士でグループがある場合,グループを明確にする.
以下の図では上述と同様にInstagramの画面における要素同士のグルーピングを抽出したものである.


\section{アルゴリズムの構築}
次にこの要素,要素同士の優先度,要素のグルーピングから1画面のUIを構成するアルゴリズムを検討した.

要素を優先度の高い順に画面内に縦に並べていくのが基本.
グループは一つの要素としてカウントする.(優先度は中の要素の優先度の足し合わせ)
要素の中にはユーザが操作するもの,見るだけのものがあり,操作するものは全体に並ぶと良い.
Groupが入れ子になる場合はHstack, Vstackを交互にすることによって大体のUIを構成できる.
画面内の要素の優先度の基準を10とし,それを超える場合はスクロールするようにする.
これで大体いい感じになる.Padding等は適切につけていく.
優先度が上がる度に文字は大きく,太く,ボタンもただの青文字ではなく,borderがついたり,塗り領域が増えたりとしていく.

