% 独自のコマンド

% ■ アブストラクト
%  \begin{jabstract} 〜 \end{jabstract}  :日本語のアブストラクト
%  \begin{eabstract} 〜 \end{eabstract}  :英語のアブストラクト

% ■ 謝辞
%  \begin{acknowledgment} 〜 \end{acknowledgment}

% ■ 文献リスト
%  \begin{bib}[100] 〜 \end{bib}


\newif\ifjapanese

\japanesetrue  % 論文全体を日本語で書く(英語で書くならコメントアウト)

\ifjapanese
  \documentclass[a4j,twoside,openright,11pt]{jreport} % 両面印刷の場合。余白を綴じ側に作って右起こし。
  % \usepackage[backend=bibtex]{biblatex}
  %\documentclass[a4j,11pt]{jreport}                  % 片面印刷の場合。
  \renewcommand{\bibname}{参考文献}
  \newcommand{\acknowledgmentname}{謝辞}
\else
  \documentclass[a4paper,11pt]{report}
  \newcommand{\acknowledgmentname}{Acknowledgment}
\fi
\usepackage{thesis}
\usepackage{ascmac}
\usepackage[dvipdfmx]{graphicx}
\usepackage{multirow}
\usepackage{url}
\usepackage{lscape}
%\bibliographystyle{jplain}
\bibliographystyle{junsrt}

\bindermode  % バインダー用余白設定

% 日本語情報(必要なら)
\jclass  {卒業論文}                             % 論文種別
\jtitle    {ユーザインタフェースの設計をサポートするノーデザインツールの研究}    % タイトル。改行する場合は\\を入れる
\juniv    {慶應義塾大学}                  % 大学名
\jfaculty  {環境情報学部}               % 学部、学科
\jauthor  {尾崎 正和}                       % 著者
\jhyear  {3}                                   % 平成○年度
\jsyear  {2021}                                 % 西暦○年度
\jkeyword  {UI, ノーデザイン, ノーコード, プロダクト開発, UI自動生成, インターフェースビルダー, 人間中心設計}     % 論文のキーワード
\jproject{全世界インタフェースデザイン (増井研究会)} %プロジェクト名
\jdate{2022年1月}

% 英語情報(必要なら)
\eclass  {Graduation Thesis}                            % 論文種別
\etitle    {Research on the realization of "no-design" tools that go beyond no-codes}      % タイトル。改行する場合は\\を入れる
\euniv  {Keio University}                             % 大学名
\efaculty  {Faculty of Environment and Information Studies}  % 学部、学科
\eauthor  {Masakaz Ozaki}                           % 著者
\eyear  {2021}                                        % 西暦○年度
\ekeyword  { User Interface, No-design, No-code, Product Development, UI Generation, Interface Builder, Human Centered Design }          % 論文のキーワード
\eproject{Masui Lab}                 %プロジェクト名
\edate{January 2022}





\begin{document}

\ifjapanese
  \jmaketitle    % 表紙(日本語)
\else
  \emaketitle    % 表紙(英語)
\fi


\begin{jabstract}

スマートフォンやパーソナルコンピュータの急激な普及につれ,様々な応用ソフト(アプリケーション)が開発され人々が毎日利用している.近年ではアプリケーション開発において十分な開発時間,人員,技術が確保することが難しく質の担保されたアプリケーションを設計し開発することが難しくなっている.そして個人開発者,スタートアップ企業,アイディアはあるが実装する力のない個人などではさらにこれを行うことが難しくなる.

その流れに伴い「ノーコード」と呼ばれるプログラミングやコンピュータサイエンスに精通していない人でも簡単にアプリケーションを開発できるサービスが登場した.しかしながら現在のノーコードにおいてもアプリケーションのユーザインタフェースデザインはユーザが行うか,テンプレートを利用する他ないためユーザインタフェースに精通していないユーザでは自由度の高く質の担保されたユーザインタフェースのアプリケーションを開発するのは困難である.

そこで本研究では,本来大規模な開発でしか行われない専門のユーザインタフェースデザインを自動化し,ユーザインタフェースに関する専門的な知識がなくても容易にユーザインタフェースを開発できるようにするためのシステムの開発をおこなった.ユーザインタフェースの設計の方法については確立された手法はなくデザイナのセンスと経験に委ねられていた部分が多かったが,既存のユーザインタフェースを分析していくことでユーザインタフェースは画面に表示されている要素の数と種類,各要素のグルーピング,画面内での要素がもつ優先度が決定されればインターフェースを制作できるのではないかという仮説を立てた.その仮説に基づきiOS向けのネイティブアプリケーションのユーザインタフェースを専門的知識なしに制作できるシステムを開発し有用性を明らかにした.
\end{jabstract}


\begin{eabstract}
With the rapid spread of smart phones and personal computers, a variety of application software have been developed and are used by people every day. In recent years, it has become difficult to secure sufficient development time, manpower, and technology to design and develop applications with guaranteed quality. This is even more difficult for individual developers, start-up companies, and individuals who have ideas but lack the ability to implement them.

In response to this trend, a service called "no-code" has emerged, which allows people who are not familiar with programming or computer science to easily develop applications. However, even with the current "no-code" service, the user interface design of the application can only be done by the user or by using a template, making it difficult for users who are not familiar with user interfaces to develop applications with a high degree of freedom and guaranteed quality. Therefore, it is difficult for users who are not familiar with user interfaces to develop applications with high flexibility and quality.

Therefore, in this research, we developed a system to automate specialized user interface design, which is normally done only in large-scale development, so that users without specialized knowledge of user interfaces can easily develop user interfaces. There is no established method for designing user interfaces, and much of the work has been left to the designer's sense and experience. However, by analyzing existing user interfaces, we hypothesized that a user interface can be created if the number and type of elements displayed on the screen, the grouping of each element, and the priority of each element within the screen are determined. Based on this hypothesis, we developed a system that can create user interfaces for native applications for iOS without specialized knowledge, and clarified its usefulness.

\end{eabstract}
  % アブストラクト。要独自コマンド、include先参照のこと

\tableofcontents  % 目次
\listoffigures    % 表目次
\listoftables    % 図目次

\pagenumbering{arabic}

\chapter{序論}
\label{chap:introduction}

\section{背景}

アプリケーション開発が近年盛んになってる。

デザインの重要性を述べる
%Goodpatchを持ってくる。シリコンバレーは共同創業者にデザイナがいてユーザの体験をきちんと設計していた

そしてデザインは表層のデザインとされがちであるが本質は奥深くにあると述べる
%Goodpatchのダイアグラムを引用
本システムでは勘違いされやすい表層のデザインを自動化することで
だが

\subsection{大規模開発に置けるユーザインタフェースデザイン}
\subsection{個人開発に置けるユーザインタフェースデザイン}
実開発で費やされるUIデザインの時間について述べる

(1)現在は大規模な開発でしか行われていない専門性の高いユーザインタフェースデザインのハードルを下げること,(2)表層のデザインにとらわれずに奥深くのデザインに注力することができる手法を開発することが求められる.

\section{目的}
(1),(2) を満たすための,自動化で要素,優先度,グルーピングを使ったものを開発した.そして,開発したシステムの有用性を示した.
%
\section{本論文の構成}

本論文の構成を示す.

第\ref{chap:introduction}章では本研究の背景について述べた.第\ref{chap:prevresearch}章では関連研究と諸概念を整理する.第\ref{chap:pulsewave}章では要素,優先度,グルーピングからUIを自動生成する手法を提案し,第\ref{chap:pulsewave}章ではその手法をもとにiOS向けネイティブアプリケーションのコードを生成するシステムを提案する.そして,それらの有効性を示す.最後に,第\ref{chap:conclusion}章の結論では本研究を総括し,考察と展望を述べる.付録として,本研究で行った実験で得られたデータを添付する.
  % 本文1
\chapter{関連研究と諸概念の整理}
\label{chap:prevresearch}

\section{ノーコード}
既存のノーコードの例をいくつか上げ,STUDIO,Glideなどそれについての説明と問題点を挙げる.

\subsection{STUDIO}

\subsection{Glide}


\section{大規模開発に置けるユーザインタフェースデザイン}
\section{個人開発に置けるユーザインタフェースデザイン}
%
%User Experienceという言葉は,1995年にはAppleのヒューマンインタラクショングループで使われており\cite{norman1995},ヒューマンインタフェースやユーザビリティといった言葉ではカバーできない工業デザインのグラフィック,インタフェース,物理的なインタラクション,マニュアルなど人がシステムを利用する際のあらゆる側面を含む言葉として発明したとNormanがインタビューで述べている\cite{normaninterview}.しかしその後,UXという言葉の普及に伴ってあらゆる事象に使われるようになり本来の意味が不明瞭になりつつあった.そこでロトらによってUX白書でUXの定義が定められた.UX白書では,UXとは``「一般的な意味における経験」とは異なり,システムと出会うことにおける経験''であるとしている\cite{uxwhitepaper}.
%
%UX白書によるUXの定義に「システムと出会うこと」という文言が含まれている事からわかるとおり,UXは製品の設計やシステム自体についての概念ではなくユーザと製品との間に生まれるものであり,前述のSQuaREの利用時品質に関連が深いことがわかる.また,UX白書ではUXに影響する要素(factor)として,ユーザとシステムに加えて文脈を挙げている\cite{uxwhitepaper}.UXはこれらの3つの掛け合わせで表現されるものでどれかが変化すればUXも変化するといえる.
%
%UXは製品が最終的にユーザにどのように受け取られるかという重要な概念ではあるが,一方で製品設計においては「UXを向上させる」ということが必ずしも正しいとは限らない.UXはシステムに対してユーザの数×文脈の数だけ存在するためそれら全てを向上させることは不可能だからである.実ユーザのUXを考慮して,ビジネス上の目標あるいは技術上の制約に沿うように製品設計に落とし込み,その上で製品品質を向上させる必要がある.

\section{機械学習を用いたUIの自動生成}
機械学習使うと次同じインプットでやった時にも同じのが出てくると限らないので適していない
\subsection{GANを用いた自動生成}

\subsection{GPT-3を用いた自動生成}


\section{問題の所在}
  % 本文2
\chapter{ノーデザイン: 要素,優先度,グルーピングの情報からのUI自動生成手法の提案}
\label{chap:auto-gen}

本章では,後述のiOSネイティブアプリケーションのインタフェース自動生成システムの開発に先立ち,既存のインタフェースの分析から導きさした要素,優先度,グルーピングによりインタフェースの構築が行える法則について検討した.自動生成を行うアルゴリズムの提案行う.
\section{既存インタフェースの分析}
既存のiOSネイティブアプリケーションのUIの分析をおこなっていく.
画面内でのUIは画面遷移を司り,画面内の体験には直接影響を与えないものが存在する.
わかりやすくするために,画面遷移を司るUIとそうでないUIについて分類して考えていく.
\subsection{画面遷移を司るUI}
HIGのhierarchical Navigation,
FlatNavigation
Content-Driven or Experience-Driven Navigation
大体こうなってる.


今回の自動生成手法の提案ではこのような画面遷移を司るUIは切り離し,画面内のコンテンツのUIに注目して考えいく.
実際のSwiftUIのコードでも画面遷移を司るUI,画面のタイトルとコンテンツは別で実装されて画面遷移の上にコンテンツが乗る形式になる.のでこの方針は問題ない


\subsection{画面の内容を表示するUI}
前述では画面内のUIについて画面の内容を表示するUIを画面遷移を司るUIが内包していると述べた.本セクションでは画面の内容を表示するUIのレイアウトは主に画面内に表示する要素(UIパーツ),その要素同士の優先度,そして要素同士のグルーピングの3つのルールに則って構成されており,そのルールさえ示せば画面を構成できるのではないかという仮説を立てた.次にレイアウトを構成する3つのルールについて述べる.
\subsubsection{要素}
画面内にどの要素がいくつ存在するのかを定める必要がある.
以下の図では既存のInstagramの画面における要素の抽出をおこなった.
\subsubsection{優先度}
画面内の要素の種類と個数がわかったら次に画面内でその要素の重要度を明確にする.
例えば,SNSサービスであっても写真をメインとするInstagramと文字をメインとするTwitterでは文字と画像の優先度の比重が変わってくる.
以下の図ではInstagramの画面における要素同士の優先度を抽出したものである.

\subsubsection{グルーピング}
要素,優先度に加えて各要素同士でグループがある場合,グループを明確にする.
以下の図では上述と同様にInstagramの画面における要素同士のグルーピングを抽出したものである.


\section{アルゴリズムの構築}
次にこの要素,要素同士の優先度,要素のグルーピングから1画面のUIを構成するアルゴリズムを検討した.

要素を優先度の高い順に画面内に縦に並べていくのが基本.
グループは一つの要素としてカウントする.(優先度は中の要素の優先度の足し合わせ)
要素の中にはユーザが操作するもの,見るだけのものがあり,操作するものは全体に並ぶと良い.
Groupが入れ子になる場合はHstack, Vstackを交互にすることによって大体のUIを構成できる.
画面内の要素の優先度の基準を10とし,それを超える場合はスクロールするようにする.
これで大体いい感じになる.Padding等は適切につけていく.
優先度が上がる度に文字は大きく,太く,ボタンもただの青文字ではなく,borderがついたり,塗り領域が増えたりとしていく.

  % 本文3
\chapter{iOSネイティブアプリケーションのインタフェース自動生成システムの開発}
\label{chap:impl}
前述の既存UIの分析と分析から導き出したアルゴリズムをもとにiOSのネイティブアプリケーションのインタフェース自動生成システムの開発を行った.

また,普段からiOSのネイティブアプリケーションを個人開発している14-17歳の男女を被験者とした本システムの有用性をはかる実験を行い,有用性を示すことができた.

本システムの設計は以下のとおりである.
\section{システムの設計・開発}
要素,優先度,グルーピング情報からの画面のコンテンツUI自動生成システムのプロトタイプとして主要な部分のみを実装した.本研究において画面のコンテンツUIとは実際画面に表示されるUIのうち,画面遷移を司どるUI,画面のタイトル表示を除いたものを指す.
本研究のUI自動生成において重要なのは画面内の要素,優先度,グルーピング,の情報のみでUIを自動生成できることである.この点を実現するためのシステムを開発した.

\subsection{基本的なレイアウトアルゴリズム}
前述のアルゴリズムより,基本的には要素を縦に順に配列する.グループも一要素として扱いレイアウトを行う.グループ内のレイアウトについては御述のグルーピングセクションで行う.
基本的な優先度

\subsection{優先度}


\subsection{グルーピング}
\subsubsection{グルーピングの入れ子構造}

\subsection{表示レイアウトアルゴリズム}

\section{システムの有効性についての実験}

\subsection{対象と手続き}

\subsection{結果と考察}  % 本文4
\chapter{結論}
\label{chap:conclusion}

\section{システムの提案}

\section{今後の課題}
\subsection{配色の自動生成}
\subsection{スライダーUIによる表層デザインの自由化}
\subsection{統合的なシステムとしての展望}
  % 本文5

\begin{acknowledgment}

本論文の作成にあたり,研究室聴講時代から終始適切な助言を賜り,また丁寧に指導してくださった増井俊之教授に深く感謝申し上げます.

同輩である孫正義育英財団正財団生の佐々木雄司氏には気の置ける友人として,日々的確で新しい意見をいただいた他,TeXの環境構築にも多大なるご支援をいただきました.感謝申し上げます.

\end{acknowledgment}
  % 謝辞。要独自コマンド、include先参照のこと

\begin{bib}[99]

\bibliography{main}

\end{bib}
  % 参考文献。要独自コマンド、include先参照のこと
\appendix
\chapter{UI自動生成システムユーザテスト アンケート質問紙}
\begin{enumerate}
  \item 年齢
  \item 性別(Biological)
  \\選択肢
 \begin{itemize}
  \item 男性
  \item 女性
  \item その他
  \item 回答したくない
\end{itemize}
\item 普段,どれぐらいパソコンを使っていますか
\\選択肢
\begin{itemize}
	\item 1. ほとんど触らない
	\item 2. 得意とは言えないが,調べもの程度になら使える
	\item 3. コンピュータは日頃から利用していて,プログラミングでものづくりもおこなっている
\end{itemize}
\item 普段アプリ開発において,デザインとはどのようなことをおこなっていますか.
\item 今回の自動システムを使った率直な感想をお願いします.
\item 優先度を調整して思い通りのUIに近づきましたか.
\item 普段UIを考える,実装する上で大変だと思うことは何ですか.
\item 全体を通しての感想や意見などなんでも記入してください.
\end{enumerate}

\chapter{UI自動生成システムユーザテスト アンケート回答}
\section{選択問題回答(問1-3)}
\begin{table}[htbp]
\centering
\scalebox{1.0}{
\begin{tabular}{llllllllllllllll}
\hline
年齢                    &性別          &普段,どれぐらいパソコンを使っていますか \\ \hline
13  &F  &3\\
15  &F  &3\\
16  &F  &3\\
17  &M  &3\\
16  &F  &3\\ \hline
\end{tabular}
}
\caption{各要素の配置係数}
\label{table:viewArrangementRatio}
\end{table}



\section{普段アプリ開発において,デザインとはどのようなことをおこなっていますか.}
\begin{itemize}
	\item 色々なモチーフを組み合わせてSketchやイラレに書き起こす
	\item テーマを決めて統一させる、パーツを置いてみる、色合いを見る
	\item 載せたい内容を一度整理する。
	\item PinterestでUIのデザインを調べたり、友だちや家族にUIのデザインの案を見てもらい、どうしたら、もっと見やすくなるかなどを聞いて、より使いやすいUIのデザイン考えています。
	\item ラフスケッチ、ワイヤーフレームを作ってみたりして、ユーザーに見せたい順番で、文字の大きさや配置を考えてどの順番で追っていくかを考える
\end{itemize}

\section{今回の自動システムを使った率直な感想をお願いします.}
\begin{itemize}
	\item すげぇ!!デザインするとき時短になりそう
	\item 自分でやらないといけないことが自動でできて嬉しかった
	\item 画期的で素晴らしいと思いました。
	\item UIのデザインを考えるとき、かなり悩むので、悩まなくてよくなるのは、とても便利だと思いました。
	\item UIパーツと優先順位の数の二つを決めるだけで実装できるようになるのが簡単で普段時間がかかる実装も試しやすかった
\end{itemize}

\section{優先度を調整して思い通りのUIに近づきましたか.}
\begin{itemize}
	\item 近づいた
	\item 近づいた
	\item 近づいた。
	\item 近づいた!!優先順位を決めるだけで大きさを勝手に調節してくれるから、デザインに悩む時間が減って実装しやすくなると思った!!
	\item 近づきました。
\end{itemize}

\section{普段UIを考える,実装する上で大変だと思うことは何ですか.}
\begin{itemize}
	\item バランス、押しやすいボタンの幅を考えたりすること
	\item パーツの場所を考える、コンセプト?を決める
	\item デザインの見栄えを考えるあまり、UIとしての本来の目的を見失いがち。
	\item どのようなUIが直感的に使えるか、どうしたら見やすくなるかを考えている。考えたデザインをどうやってダークモードとライトモードに対応させるかが大変だと思っている。
	\item AutoLayoutを調節したり、配置したりすることが大変

\end{itemize}

\section{全体を通しての感想や意見などなんでも記入してください.}
\begin{itemize}
	\item AutoLayoutが楽になりそう!みんなが使ってハッカソンで同じUIが揃うの想像したら面白い
	\item コードを書かなくて済んで嬉しい
	\item 情報の優先度と載せる情報を整理して、デザインを考える過程を楽にできるので、UI構成の効率性を図ることができると思います。
	\item UIを作るときに、オートレイアウトを設定するのが面倒くさいので、優先順位をつけるだけで見やすいUIになるのはとても画期的だと思った。
	\item 簡単に実装できるようになるのが楽しくてこれから使えるようになったら、どんどん使ってみたい!
\end{itemize}    % 付録

\end{document}
