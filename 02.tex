\chapter{関連研究と諸概念の整理}
\label{chap:prevresearch}

\section{ユーザインタフェースのデザイン手法}

%なんか内容がびみょういな.知りたいのはそういうことではない

\subsection{大規模開発に置けるユーザインタフェースデザイン}
\subsection{個人開発に置けるユーザインタフェースデザイン}

\section{ノーコードでのシステム開発}
既存のノーコードの例をいくつか上げ,STUDIO,Glideなどそれについての説明と問題点を挙げる.

\subsection{STUDIO}

\subsection{Glide}

%
%User Experienceという言葉は,1995年にはAppleのヒューマンインタラクショングループで使われており\cite{norman1995},ヒューマンインタフェースやユーザビリティといった言葉ではカバーできない工業デザインのグラフィック,インタフェース,物理的なインタラクション,マニュアルなど人がシステムを利用する際のあらゆる側面を含む言葉として発明したとNormanがインタビューで述べている\cite{normaninterview}.しかしその後,UXという言葉の普及に伴ってあらゆる事象に使われるようになり本来の意味が不明瞭になりつつあった.そこでロトらによってUX白書でUXの定義が定められた.UX白書では,UXとは``「一般的な意味における経験」とは異なり,システムと出会うことにおける経験''であるとしている\cite{uxwhitepaper}.
%
%UX白書によるUXの定義に「システムと出会うこと」という文言が含まれている事からわかるとおり,UXは製品の設計やシステム自体についての概念ではなくユーザと製品との間に生まれるものであり,前述のSQuaREの利用時品質に関連が深いことがわかる.また,UX白書ではUXに影響する要素(factor)として,ユーザとシステムに加えて文脈を挙げている\cite{uxwhitepaper}.UXはこれらの3つの掛け合わせで表現されるものでどれかが変化すればUXも変化するといえる.
%
%UXは製品が最終的にユーザにどのように受け取られるかという重要な概念ではあるが,一方で製品設計においては「UXを向上させる」ということが必ずしも正しいとは限らない.UXはシステムに対してユーザの数×文脈の数だけ存在するためそれら全てを向上させることは不可能だからである.実ユーザのUXを考慮して,ビジネス上の目標あるいは技術上の制約に沿うように製品設計に落とし込み,その上で製品品質を向上させる必要がある.

\section{デザインの自動化}
\subsection{機械学習を用いたUIの自動生成}
機械学習使うと次同じインプットでやった時にも同じのが出てくると限らないので適していない
\subsubsection{GANを用いた自動生成}

\subsubsection{GPT-3を用いた自動生成}

\section{問題の所在}

\begin{itemize}
	\item デザインを表層のデザインだけだと誤解している %表層のデザインしかしないシステムを作るので言葉が足りない
	\item 既存のノーコードではデザインは自分で行う又はテンプレートから選んだデザインをそのまま使用する必要がある
	\item 機械学習を用いた自動生成では一意性が担保できない.
	\item GANで生成したUIは画像なのでアプリとして機能しない.
\end{itemize}
また,スマートフォンのUIは2007年の初代iPhoneから,パソコンについては1984年の初代Macintoshから基本的なUIは全く変わっていないにもかかわらず,UIの生成の法則性の研究はあまり行われてこなかった.


以上のことから表層のデザインを自動化し,専門知識のない人でも
