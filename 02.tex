\chapter{関連研究と諸概念の整理}
\label{chap:prevresearch}

\section{ユーザインタフェースのデザイン手法}

%なんか内容がびみょういな.知りたいのはそういうことではない

\subsection{大規模開発に置けるユーザインタフェースデザイン}
\subsection{個人開発に置けるユーザインタフェースデザイン}

\section{ノーコードでのシステム開発}
大規模な開発を行うリソースがない企業や専門知識をあまり持たない個人でもアプリケーションやウェブサイトを作れるように近年,ノーコードと呼ばれるサービスが登場した.ノーコードとは文字通り,コーディングを行わずにアプリケーションやウェブサイトを作れるサービスである.

全体を通して,コードを書かないのでコードを書けばできることができない.


ここでは既存のノーコードのサービスの例をいくつかあげ,特徴と課題点を挙げていく.
\subsection{STUDIO}
STUDIOはSTUDIO株式会社が運営する

デザインの自由度が高いのがウリ.自由度の高いノーコードってイラレみたいで結局表層のビジュアルデザインに目を向けがち.
web front-endの専門知識はなくても行えるが,デザインの専門知識は求められる.
自由度の高いビジュアルデザインを売りにしているが,細かなインタラクションなどを追加でソースコードを書くことで補えないのでSTUDIOのできる範囲内で行わなければならない.
素人ではなくある程度知識のある人が使うものであるにもかかわらず,STUDIOで制作したwebサイトは一目でわかってしまうため,コードを書けないと自分で宣言しているようなものである.

\subsection{Glide}
Google Spread Sheetsなどをデータベースとして扱い,テンプレートベースの簡易なアプリを作れる.
スケーラビリティが終わっている
テンプレートベース
社内ツールの自動化や小規模のもの向けでこれで作ったものをプロダクトとして出すことは難しい


\subsection{Microsoft Power Apps}
Glideと同じような感じであるが,若干のスケーラビリティがる.
同じくビジネスツールなのでそれ自体をプロダクトにすることは難しい

\section{デザインの自動化}
デザインの専門知識がない人でもテンプレートに頼らず適切のUIを制作できるようにUIの自動生成を行う研究が行われている
\subsection{機械学習を用いたUIの自動生成}
機械学習使うと次同じインプットでやった時にも同じのが出てくると限らないので適していない
\subsubsection{GANを用いた自動生成}

\subsubsection{GPT-3を用いた自動生成}

\section{問題の所在}

\begin{itemize}
	\item デザインを表層のデザインだけだと誤解している %表層のデザインしかしないシステムを作るので言葉が足りない
	\item 既存のノーコードではデザインは自分で行う又はテンプレートから選んだデザインをそのまま使用する必要がある
	\item 機械学習を用いた自動生成では一意性が担保できない.
	\item GANで生成したUIは画像なのでアプリとして機能しない.
\end{itemize}
また,スマートフォンのUIは2007年の初代iPhoneから,パソコンについては1984年の初代Macintoshから基本的なUIは全く変わっていないにもかかわらず,UIの生成の法則性の研究はあまり行われてこなかった.


以上のことから表層のデザインを自動化し,専門知識のない人でも
